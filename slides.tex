\documentclass[pdf, hyperref={unicode}, aspectratio=169]{beamer}

\usepackage{styles}

\title{Создание презентации к защите ВКР}

\subtitle{Выпускная квалификационная работа бакалавра}

\pdfstringdefDisableCommands{
  \def\\{}
  \def\,{}
  \def\textbf#1{<#1>}
}

\author[Екатерина Павловна Самсонова]
{
  \textbf{Студент группы М8О-407Б-19:} Екатерина Павловна Самсонова\\
  \ \textbf{Научный руководитель:} д.ф.-м.н. проф. каф. 806 О.\,Ю.\,Табити
  % Обратите внимание на пробел в начале строки
}

\institute[Московский авиационный институт]
{
  Московский авиационный институт (национальный исследовательский университет)\\
  Институт № 8 «Компьютерные науки и прикладная математика»\\
  Кафедра № 806 «Вычислительная математика и программирование» 
}

\date{Москва --- \the\year}

\logo{\includegraphics[height=1cm]{img/mai}}

\begin{document}

\frame{\titlepage}

\begin{frame}
\frametitle{Актуальность темы}

\begin{itemize}
	\item Краткое описание/иллюстрация проблемы, которую пытался решить студент в своей ВКР(б), доказывающей реальность проблемы и раскрывающей суть проекта
	\item Это может быть график динамики с негативным трендом, секторная диаграмма с опасно высокой доли какой-либо компоненты, библиометрия и др. Это должен быть достоверный факт со ссылкой на источник
\end{itemize}
\end{frame}


\begin{frame}
\frametitle{Цель и задача работы}

\textbf{Цель} --- \alert{цель}

\textbf{Задачи:}
\begin{itemize}
	\item задача один
	\item задача два
	\item задача три
\end{itemize}
\end{frame}


\begin{frame}
\frametitle{Постановка задачи}

\textbf{Дано:}
\begin{itemize}
	\item Исходные данные
	\item Требования и условия
	\item Ограничения и допущения
	\item Требования к железу, ОС и сетевому окружению
	\item …
\end{itemize}

\textbf{Необходимо} (образ результата)
\end{frame}


\begin{frame}
\frametitle{Логика работы}

Перечислить основные разделы работы и указать их последовательность
\end{frame}


\begin{frame}
\frametitle{Используемые инструменты}

\begin{itemize}
	\item Языки программирования
	\item Какие методы и инструменты использовались
	\item Ссылка на репозиторий кода
	\item Ссылка на использованные данные
	\item …
\end{itemize}
\end{frame}


\begin{frame}
\frametitle{Архитектура решения, алгоритм решения задачи}

Блок-схема/Разрабатываемые модули и их взаимодействие
\end{frame}


\begin{frame}
\frametitle{Работа с данными}

\begin{itemize}
	\item Обучающая выборка
	\item Тестовая выборка
	\item Источники
	\item …
\end{itemize}
\end{frame}


\begin{frame}
\frametitle{Результаты разработки}

Скрины/Формат вывода данных
\end{frame}


\begin{frame}
\frametitle{Оценка результата}

Перечень ключевых характеристик функционирования разработки
\end{frame}


\begin{frame}
\frametitle{Описание программной разработки}

QR-код ссылки, где выложен код

\begin{figure}
\includegraphics[height=0.7\textheight]{img/qr-code}
\end{figure}
\end{frame}

\end{document}
